\documentclass[11pt,a4paper]{article}
\usepackage[T1]{fontenc}
\usepackage{inputenc}
\usepackage{authblk}

\usepackage{geometry}

\geometry{
body={6.5in, 8.5in},
left=1.0in,
top=1.25in
}
  
\title{Microscopio combinado de fuerza at�mica-�ptico.}
\author[1]{M. Caldarola \thanks{caldarola@df.uba.ar}}
\author[2]{C. von Bildering}
\author[2,3]{L. E. Pietrasanta}
\author[1,3]{A. V. Bragas}
\affil[1]{Laboratorio de Electr�nica Cu�ntica, DF, FCEyN, UBA}
\affil[2]{Centro de Microscop�as Avanzadas, FCEyN, UBA}
\affil[3]{IFIBA, CONICET}


\date{}

\renewcommand\Authands{ y }


\begin{document}
  \maketitle
  \thispagestyle{empty}
  \pagestyle{empty}

La combinaci�n de la microscop�a �ptica y la microscop�a/espectroscop�a de fuerza at�mica, permite simult�neamente localizar, cuantificar y manipular interacciones moleculares relevantes al complejo funcionamiento de una c�lula viva. En este trabajo presentamos un microscopio de fuerza at�mica (AFM) combinado con un microscopio �ptico que permite realizar im�genes topogr�ficas y �pticas de fluorescencia simult�neamente. Estas t�cnicas proveen ventajas complementarias: mientras que la fluorescencia ofrece alta resoluci�n temporal para localizaci�n molecular espec�fica, el AFM aporta informaci�n topogr�fica en escala nanom�trica y permite la detecci�n de fuerzas de interacci�n molecular a nivel de mol�cula �nica.


\end{document}