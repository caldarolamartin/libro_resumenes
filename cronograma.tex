%\begin{landscape}

\chapter{Programa}

% These set the width of a day and the height of an hour.
%\newcommand*\daywidth{4.5cm}
%\newcommand*\hourheight{3em}

\newcommand*\daywidth{3.4cm}
\newcommand*\hourheight{3em}

% The entry style will have two options:
% * the first option sets how many hours the entry will be (i.e. its height);
% * the second option sets how many overlapping entries there are (thus
%   determining the width).
\tikzset{entry/.style 2 args={
    xshift=(0.5334em+0.8pt)/2,
    draw,
    line width=1pt,
    font=\sffamily,
    rectangle,
    rounded corners,
    fill=white,
    anchor=north west,
    inner sep=0.1em,
    text width={\daywidth/#2-0.85em},
    minimum height=#1*\hourheight,
    align=center
}}

% Start the picture and set the x coordinate to correspond to days and the y
% coordinate to correspond to hours (y should point downwards).
\begin{tikzpicture}[y=-\hourheight,x=\daywidth]

    % First print a list of times.
    \foreach \time/\ustime in {8/8am,9/9am,10/10am,11/11am,12/12pm,13/1pm,14/2pm,15/3pm,16/4pm,17/5pm,18/6pm,19/7pm}
        \node[anchor=north east] at (1,\time) {\time~hs};

    \foreach \time/\ustime in {8.30/8.30am,9.30/9.30am,10.30/10.30am,11.30/11.30am,12.30/12.30am,13.30/1.30pm,14.30/2.30pm,15.30/3.30pm,16.30/4.30pm,17.30/5.30pm,18.30/6.30pm}
        \node[anchor=north east] at (1,\time+0.2) {\time~hs};        
        
    % Draw some day dividers.
    \draw (1,7) -- (1,19.5);
    %\draw (2,7.5) -- (2,19.5);
    \draw (3,7) -- (3,19.5);
    %\draw (4,7.5) -- (4,19.5);
    \draw (5,7) -- (5,19.5);

    % Start Monday.
    \node[anchor=north] at (2,6.8) {\large EEOF};

    \node[anchor=north] at (1.5,7.5) {\textbf{Lunes 20}};
    % Write the entries. Note that the x coordinate is 1 (for Monday) plus an
    % appropriate amount of shifting. The y coordinate is simply the starting
    % time.
    \node[entry={0.5}{1}] at (1,8.5) {Inscripci\'on};
    \node[entry={1.5}{1}] at (1,9) {Fernando Stefani};
    \node[entry={0.5}{0.5}, pattern= north west lines, pattern color=black!30] at (1,10.5) {Caf\'e};
    \node[entry={1.5}{1}] at (1,11) {Fernando Stefani};
    \node[entry={1.5}{0.5}, pattern= north west lines, pattern color=black!30] at (1,12.5) {Almuerzo};
    \node[entry={1.5}{1}] at (1,14) {Andrea Bragas};
    \node[entry={0.5}{0.5}, pattern= north west lines, pattern color=black!30] at (1,15.5) {Caf\'e};
    \node[entry={1.5}{1}] at (1,16) {Andrea Bragas};
    \node[entry={1}{0.5}] at (1,17.5) {Visita a laboratorios};
  
    
    % The same for Tuesday.
    \node[anchor=north] at (2.5,7.5) {\textbf{Martes 21}};
    \node[entry={1.5}{1}] at (2,9) {Stefan Maier};
    \node[entry={1.5}{1}] at (2,11) {Stefan Maier};
    \node[entry={1.5}{1}] at (2,14) {Rajesh Menon};
    \node[entry={1.5}{1}] at (2,16) {Rajesh Menon};
    %\node[entry={1}{1}] at (2,17.5) {Visita a laboratorios};
    
    \node[anchor=north] at (4,6.8) {\large TOpFot};

    \node[anchor=north] at (3.5,7.5) {\textbf{Mi\'ercoles 22}};
    \node[entry={0.5}{1}] at (3,8.5) {Inscripci\'on};
    \node[entry={1}{1}] at (3,9) {Rajesh Menon};
    \node[entry={1}{1}] at (3,10) {Stefan Maier};
    \node[entry={0.5}{0.5}, pattern= north west lines, pattern color=black!30] at (3,11) {Caf\'e};
    \node[entry={1}{1}] at (3,11.5) {Mar\'ia Jos\'e Galante};
    \node[entry={2}{0.5}, pattern= north west lines, pattern color=black!30] at (3,12.5) {Almuerzo};
    \node[entry={1}{1}] at (3,14.5) {Andr\'es Rieznik};
    \node[entry={1}{1}] at (3,15.5) {Galo Soler-Illia};
    \node[entry={0.5}{0.5}, pattern= north west lines, pattern color=black!30] at (3,16.5) {Caf\'e};
    \node[entry={1}{1}] at (3,17) {Presentaci\'on posters};
    \node[entry={1.5}{1}] at (3,18) {Sesi\'on posters};
    \node[entry={0.5}{1},draw=none] at (3,20) {\'Agape};
    

    \node[anchor=north] at (4.5,7.5) {\textbf{Jueves 23}};
    \node[entry={1}{1}] at (4,9) {Carlos Saavedra};
    \node[entry={1}{1}] at (4,10) {Ariel Levenson};
    \node[entry={1}{1}] at (4,11.5) {F\'elix Requejo};
    \node[entry={2}{1}] at (4,14.5) {\footnotesize Mesa redonda: \normalsize Redes nacionales de nanotecnolog\'ia};
    \node[entry={0.5}{1},draw=none] at (4,17.5) {Asamblea};

    \end{tikzpicture}

\subsection*{Lunes 20}

\subsubsection*{Fernando Stefani: ¿Qué es la plasmónica y para qué sirve?}

Lo que hoy conocemos como Plasmonica es un campo interdisciplinario
del conocimiento y la tecnología que concentra un de las mayores
actividades de investigación actual. Aunque el término Plasmónica es
relativamente reciente, algunos aspectos de la plasmónica pueden
rastrearse hasta varios siglos atrás. Estas clases tienen como
objetivo clarificar qué es exactamente la plasmónica y explicar como
se ha desarrollado desde distintas disciplinas. Se definirá lo que es
un plasmón, un polaritón, y se describirán las principales propiedades
y aplicaciones de los plasmones-polaritones de superficie.

\subsubsection*{Andrea Bragas: La plasmónica en el sensado y las microscopías}

En estas charlas, describiremos desde los mecanismos básicos relevantes hasta las aplicaciones de la plasmónica al sensado ultrasensible y la microscopía con resolución nanométrica.  Abordaremos el tema de la intensificación Raman en superficies y en puntas, el diseño de sondas plasmónicas, los mecanismos de transferencia de energía en sistemas híbridos y las vibraciones en nanopartículas plasmónicas.

\subsection*{Martes 21}

\subsubsection*{Stefan Maier: Nanoplasmonics: Fundamentals and Applications}

This short course will provide an introduction and an overview of the current
state of the art in the research field of nanoplasmonics, one of the
cornerstones of nanophotonics. The fundamentals of light localisation on the
nanoscale, below the diffraction limit, will be discussed, in addition to
applications in photovoltaics, sensing, highly integrated waveguiding, and
quantum optics.

    \subsubsection*{Rajesh Menon: Circumventing the far-field diffraction limit
in optical nanopatterning and imaging}

A technique for creating deterministic structural complexity is essential to
achieve high functionality at the
nanoscale, whether in electronics, photonics, or molecular biology.
Scanning-electron-beam lithography (SEBL) is
the most widely used method in research, but it has a number of drawbacks. SEBL
tends to be slow, expensive,
prone to placement errors, and not compatible with organics and biological
material. Ideally one would prefer to
employ benign photons in the visible or near IR range for such patterning.
However, the so-called far-field
diffraction barrier (first realized by Abb\'e) limits the smallest feature
achievable by wavelength, $\lambda$ to $\sim \lambda / 4$. The
spacing between nearest-neighbor patterns cannot be smaller than $\sim \lambda /
4$.
In this presentation, I will review the approaches for nanoscale imaging in
fluorescence microscopy.
Furthermore, I will describe two distinct approaches being investigated in my
laboratory to circumvent the far-field
diffraction limit and thereby enable optical nanopatterning with feature
spacings smaller than $\lambda / 4$.

% \vspace{0.5cm}
% 
% \textbf{References}
% 
% [1] T. L. Andrew, H. Y. Tsai, and R. Menon, ``Confining light to deep
% subwavelength dimensions to enable optical nanopatterning'', Science324(5929),
% 917-921 (2009).
% 
% [2] R. Menon and H. I. Smith, ``Absorbance-modulation optical lithography'',J.
% Opt. Soc. Am. A 23(9), 2290-2294 (2006).
% 
% [3] F. Masid, T. L. Andrew and R. Menon, ``Optical patterning of features with
% spacing below the far-field diffraction limit using absorbance modulation'',
% Opt. Exp. 21, 4 5209-5214 (2013). 
% 
% [4] H-Y. Tsai, S. W. Thomas, III and R. Menon, ``Scanning optical
% nanoscopy with
% optically confined probe'', Opt. Exp. 18(15), 16015 (2010).
% [5] N. Brimhall, T. L. Andrew, R. V. Manthena and R. Menon, ``Breaking the
% far-field diffraction limit in optical nanopatterning via repeated
% photochemical and electrochemical transitions in photochromic
% molecules'',Phys. 
% Rev. Lett. 107, 205501 (2011).
% 
% [6] P. Cantu, N. Brimhall, T. L. Andrew, R. Castagna, C. Bertarelli and R.
% Menon, ``Subwavelength nanopatterning of photochromic diarylethene
% films'', Appl. Phys. Lett. 100, 183103 (2012).\\
    
\subsection*{Mi\'ercoles 22}

\subsubsection*{Rajesh Menon: \textit{Optical nanopatterning and nanoscale
imaging: going beyond the far-field diffraction barrier}}

\subsubsection*{Stefan Maier: \textit{New frontiers in nanoplasmonics: from
fundamental nanocavity design to applications in sensing, solar light
harvesting, and nonlinear nano-optics}}

Nanoplasmonics allows the confinement of light on length scales far below the
wavelength. This talk will describe some of the current fundamental frontiers of
the field, such as nonlocal effects on nanometre length scales and
transformation optics design of broadband light harvesting nanostructures. The
second part will focus on applications in nonlinear light generation with high
efficiency, multi-spectral biosensing, and photovoltaics.

\subsubsection*{Mar\'ia Jos\'e Galante: Azo-crom\'oforos como alternativa para
la construcci\'on y manipulaci\'on de nanoestructuras}

We will describe the synthesis and characterization of polymers modified with
the addition of azobenzene moieties in their structure. The possible application
of these materials in optics and photonic devices will be evaluated.

A general scope on the resulting motion of azobenzene isomerization in polymers
will be given, showing that the scale range of these movements goes from little
reorientation of the azobenzene group to massive motion of polymeric material.

The lecture will be divided in three main sections, resultant of the analysis of
the three types of motions associated to the photoinduced isomerization process
of the azobenzene: the chromophore motion at the first level, the nano-domain
motion at the second level, and the third type of motion at an even larger
scale; it can be called macroscopic motion.

\subsubsection*{Andr\'es Rieznik: \textit{Red Federal de Fibras \'Opticas}}

El gobierno nacional est\'a construyendo a trav\'es de la empresa ARSAT una Red
Federal de Fibras \'Opticas (REFEFO) que cubrir\'a todo el territorio nacional
con m\'as de 50.000 kms de cables de fibra. En el marco del Plan Estrat\'egico
Argentina Conectada, la REFEFO cumplir\'a la funci\'on de columna vertebral para
el transporte de datos. Ser\'a una red de cables de fibra y equipos de
transmisi\'on de \'ultima tecnolog\'ia que cubrir\'a todo el territorio nacional
y ser\'a capaz de transportar todos los datos de larga distancia que el pa\'is
requiera. Describiremos este proyecto desde un punto de vista t\'ecnico y
centr\'andonos en el modelaje f\'isico de la red: su diseño y tipos de
tecnolog\'ia que se utilizar\'an (IP/MPLS/DWDM). Presentaremos el proyecto de
software que estamos desarrollando para simular num\'ericamente el desempeño de
la REFEFO, el proyecto FOP-ARSAT: ecuaciones relevantes, dispositivos modelados,
efectos f\'isicos considerados y potencialidades en aplicaciones fot\'onicas y
optoelectr\'onicas.

\subsubsection*{Galo Soler-Illia: \textit{Nanoparticle-Mesoporous materials:
chemistry-driven fabrication of nanocomposites with tunable optical properties}}

The rational design of nanocomposite made up of metallic nanoparticles
(MNP) confined within a thin film oxide matrix holds a promise for obtaining
integrated devices. Mesoporous oxide thinfilms (MOTF) represent attractive
template matrices for the inclusion of MNP. The nano-derived properties of these
systems are due to the metal NP dimensions, confinement, interfacial effect
 and the possibilities to combine the accessibility of the mesopore system and 
he electronic or surface properties of the MOTF matrix. In addition,  opt
cal quality NP-MOTF nanocomposites present unique potential in cataysis,
electronic and photonic devices, data storage and sensors.              

The use of physical and chemical ``forces of Nature'' such as soft chemistry and
self-assembly permit to exert an accurate chemical control of MNP size,
interface and positioning. This control is essential in order to master
structure and size-derived effects such as electron transfer, Surface Plasmon
resonance (SPR), fluorescence enhancement or surface-enhanced Raman
scattering (SERS). We will present the chemical strategies leading to the
reproducible preparation of gold and silver MNP-MOTF nanocomposites by
controlled reduction or photoreduction of metal ions in contact with a
mesoporous oxide matrix. Highly controlled plasmonic and photonic structures
combining physical and chemical properties can be obtained as a consequence of
controlling the spatial positioning of a variety of nano-building blocks such as
MNP, MOTF and chemical species such as functional groups, polymers or
biomolecules in monolayers, multilayers or lithography-assisted patterns.

\subsection*{Jueves 23}

\subsubsection*{Carlos Saavedra: \textit{Pinzas \'opticas hologr\'aficas:
Control simult\'aneo de posici\'on y rotaci\'on}}

En esta charla se presentar\'a la implementaci\'on experimental de un nuevo
m\'etodo para la generaci\'on din\'amica de pinzas \'opticas m\'ultiples. El
nuevo montaje experimental permite generar cada uno de las trampas con un estado
de polarizaci\'on lineal independiente y orientaci\'on arbitraria. Tambi\'en es
posible el control simult\'aneo de rotaci\'on independiente para cada trampa. El
haz l\'aser, tanto para la generaci\'on de trampas  m\'ultiples y de control de
polarizaci\'on, ha sido modulada utilizando un \'unico cristal l\'iquido. Se
presentar\'an los resultados experimentales de desplazamiento controlado, la
orientaci\'on y la rotaci\'on de las part\'iculas birrefringentes. Adem\'as, se
presentar\'an algunas de las aplicaciones que se encuentran en estudio empleando
este sistema

\subsubsection*{Ariel Levenson: Coherent and Dynamic Nonlinear Interactions in
2D Photonic Crystal single and coupled nanocavities}

Coupling light resonantly into a nanocavity mode is a
rather difficult task, when accomplished, new avenues are open to
efficiently produce nonlinear coherent interactions. We discuss recent
results on optical bistability, excitability, symmetry breaking and
slow light in semiconductor L3 Photonic Crystal nanocavities and
coupled nanocavities.

\subsubsection*{F\'elix Requejo: \textit{Investigaciones en Nanociencia con luz
de sincrotr\'on: analisis de propiedades fundamentales en nanomateriales}}

Fundamental properties of nanoparticles (NPs) depend on simple parameters like
their morphology, size, crystal structure, and composition. Novel chemical or
physical methods based, for instance, on colloidal syntheses have been developed
a precise control on shape and size, thus allowing studies to determine the
correlation between physical, chemical and structural properties in
nanomaterials. Synchrotron-based methodologies (XAFS, XMCD, SAXS, XES, GISAXS,
XPS, etc.) allow the experimental determination of atomic distribution, size,
shape, composition, structure, with surface or bulk sensitivity. We present here
a set of experimental analysis on complex nanoparticles and nano-arrrays in
order to show the capabilities of those techniques and their unique
characteristics for a multi-technique approach on nanomaterials
characterization. In particular we present our studies on core-shell iron oxide
doped with Mo and CoxPty NPs and CoSi2 nano-arrays in silicon, characterized by
soft and hard X-rays using EXAFS, XANES, SAXS, XPS, and GISAXS.
