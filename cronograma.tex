%\begin{landscape}

\chapter{Programa}

% These set the width of a day and the height of an hour.
%\newcommand*\daywidth{4.5cm}
%\newcommand*\hourheight{3em}

\newcommand*\daywidth{3.4cm}
\newcommand*\hourheight{3em}

% The entry style will have two options:
% * the first option sets how many hours the entry will be (i.e. its height);
% * the second option sets how many overlapping entries there are (thus
%   determining the width).
\tikzset{entry/.style 2 args={
    xshift=(0.5334em+0.8pt)/2,
    draw,
    line width=1pt,
    font=\sffamily,
    rectangle,
    rounded corners,
    fill=white,
    anchor=north west,
    inner sep=0.1em,
    text width={\daywidth/#2-0.85em},
    minimum height=#1*\hourheight,
    align=center
}}

% Start the picture and set the x coordinate to correspond to days and the y
% coordinate to correspond to hours (y should point downwards).
\begin{tikzpicture}[y=-\hourheight,x=\daywidth]

    % First print a list of times.
    \foreach \time/\ustime in {8/8am,9/9am,10/10am,11/11am,12/12pm,13/1pm,14/2pm,15/3pm,16/4pm,17/5pm,18/6pm,19/7pm}
        \node[anchor=north east] at (1,\time) {\time~hs};

    \foreach \time/\ustime in {8.30/8.30am,9.30/9.30am,10.30/10.30am,11.30/11.30am,12.30/12.30am,13.30/1.30pm,14.30/2.30pm,15.30/3.30pm,16.30/4.30pm,17.30/5.30pm,18.30/6.30pm}
        \node[anchor=north east] at (1,\time+0.2) {\time~hs};        
        
    % Draw some day dividers.
    \draw (1,7) -- (1,19.5);
    %\draw (2,7.5) -- (2,19.5);
    \draw (3,7) -- (3,19.5);
    %\draw (4,7.5) -- (4,19.5);
    \draw (5,7) -- (5,19.5);

    % Start Monday.
    \node[anchor=north] at (2,6.8) {\large EEOF};

    \node[anchor=north] at (1.5,7.5) {\textbf{Lunes 20}};
    % Write the entries. Note that the x coordinate is 1 (for Monday) plus an
    % appropriate amount of shifting. The y coordinate is simply the starting
    % time.
    \node[entry={0.5}{1}] at (1,8.5) {Inscripci\'on};
    \node[entry={1.5}{1}] at (1,9) {Fernando Stefani};
    \node[entry={0.5}{0.5}, pattern= north west lines, pattern color=black!30] at (1,10.5) {Caf\'e};
    \node[entry={1.5}{1}] at (1,11) {Fernando Stefani};
    \node[entry={1.5}{0.5}, pattern= north west lines, pattern color=black!30] at (1,12.5) {Almuerzo};
    \node[entry={1.5}{1}] at (1,14) {Andrea Bragas};
    \node[entry={0.5}{0.5}, pattern= north west lines, pattern color=black!30] at (1,15.5) {Caf\'e};
    \node[entry={1.5}{1}] at (1,16) {Andrea Bragas};
    \node[entry={1}{0.5}] at (1,17.5) {Visita a laboratorios};
  
    
    % The same for Tuesday.
    \node[anchor=north] at (2.5,7.5) {\textbf{Martes 21}};
    \node[entry={1.5}{1}] at (2,9) {Stefan Maier};
    \node[entry={1.5}{1}] at (2,11) {Stefan Maier};
    \node[entry={1.5}{1}] at (2,14) {Rajesh Menon};
    \node[entry={1.5}{1}] at (2,16) {Rajesh Menon};
    %\node[entry={1}{1}] at (2,17.5) {Visita a laboratorios};
    
    \node[anchor=north] at (4,6.8) {\large TOpFot};

    \node[anchor=north] at (3.5,7.5) {\textbf{Mi\'ercoles 22}};
    \node[entry={0.5}{1}] at (3,8.5) {Inscripci\'on};
    \node[entry={1}{1}] at (3,9) {Rajesh Menon};
    \node[entry={1}{1}] at (3,10) {Stefan Maier};
    \node[entry={0.5}{0.5}, pattern= north west lines, pattern color=black!30] at (3,11) {Caf\'e};
    \node[entry={1}{1}] at (3,11.5) {Mar\'ia Jos\'e Galante};
    \node[entry={2}{0.5}, pattern= north west lines, pattern color=black!30] at (3,12.5) {Almuerzo};
    \node[entry={1}{1}] at (3,14.5) {Andr\'es Rieznik};
    \node[entry={1}{1}] at (3,15.5) {Galo Soler-Illia};
    \node[entry={0.5}{0.5}, pattern= north west lines, pattern color=black!30] at (3,16.5) {Caf\'e};
    \node[entry={1}{1}] at (3,17) {Presentaci\'on posters};
    \node[entry={1.5}{1}] at (3,18) {Sesi\'on posters};
    \node[entry={0.5}{1},draw=none] at (3,20) {\'Agape};
    

    \node[anchor=north] at (4.5,7.5) {\textbf{Jueves 23}};
    \node[entry={1}{1}] at (4,9) {Carlos Saavedra};
    \node[entry={1}{1}] at (4,10) {Ariel Levenson};
    \node[entry={1}{1}] at (4,11.5) {F\'elix Requejo};
    \node[entry={2}{1}] at (4,14.5) {\footnotesize Mesa redonda: \normalsize Redes nacionales de nanotecnolog\'ia};
    \node[entry={0.5}{1},draw=none] at (4,17.5) {Asamblea};

    \end{tikzpicture}

\subsection*{Mi\'ercoles 22}

\subsubsection*{Rajesh Menon: \textit{Optical nanopatterning and nanoscale imaging: going beyond the far-field diffraction barrier}}

\subsubsection*{Stefan Maier: \textit{New frontiers in nanoplasmonics}}

\subsubsection*{Mar\'ia Jos\'e Galante: }

\subsubsection*{Andr\'es Rieznik: \textit{Red Federal de Fibras \'Opticas}}

El gobierno nacional est\'a construyendo a trav\'es de la empresa ARSAT una Red
Federal de Fibras \'Opticas (REFEFO) que cubrir\'a todo el territorio nacional
con m\'as de 50.000 kms de cables de fibra. En el marco del Plan Estrat\'egico
Argentina Conectada, la REFEFO cumplir\'a la funci\'on de columna vertebral para
el transporte de datos. Ser\'a una red de cables de fibra y equipos de
transmisi\'on de \'ultima tecnolog\'ia que cubrir\'a todo el territorio nacional
y ser\'a capaz de transportar todos los datos de larga distancia que el pa\'is
requiera. Describiremos este proyecto desde un punto de vista t\'ecnico y
centr\'andonos en el modelaje f\'isico de la red: su dise�o y tipos de
tecnolog\'ia que se utilizar\'an (IP/MPLS/DWDM). Presentaremos el proyecto de
software que estamos desarrollando para simular num\'ericamente el desempe�o de
la REFEFO, el proyecto FOP-ARSAT: ecuaciones relevantes, dispositivos modelados,
efectos f\'isicos considerados y potencialidades en aplicaciones fot\'onicas y
optoelectr\'onicas.

\subsubsection*{Galo Soler-Illia: \textit{Nanoparticle-Mesoporous materials:
chemistry-driven fabrication of nanocomposites with tunable optical properties}}

The rational design of nanocomposite made up of metallic nanoparticles
(MNP) confined within a thin film oxide matrix holds a promise for obtaining
integrated devices. Mesoporous oxide thinfilms (MOTF) represent attractive
template matrices for the inclusion of MNP. The nano-derived properties of these systems are due to the metal NP dimensions, confinement, interfacial effect
 and the possibilities to combine the accessibility of the mesopore system and 
he electronic or surface properties of the MOTF matrix. In addition,  opt
cal quality NP-MOTF nanocomposites present unique potential in cataysis, electronic and photonic devices, data storage and sensors.              



The use of physical and chemical ``forces of Nature'' such as soft chemistry and
self-assembly permit to exert an accurate chemical control of MNP size,
interface and positioning. This control is essential in order to master
structure and size-derived effects such as electron transfer, Surface Plasmon
resonance (SPR), fluorescence enhancement or surface-enhanced Raman
scattering (SERS). We will present the chemical strategies leading to the
reproducible preparation of gold and silver MNP-MOTF nanocomposites by
controlled reduction or photoreduction of metal ions in contact with a
mesoporous oxide matrix. Highly controlled plasmonic and photonic structures
combining physical and chemical properties can be obtained as a consequence of
controlling the spatial positioning of a variety of nano-building blocks such as
MNP, MOTF and chemical species such as functional groups, polymers or
biomolecules in monolayers, multilayers or lithography-assisted patterns.

\subsection*{Jueves 23}

\subsubsection*{Carlos Saavedra: \textit{Pinzas \'opticas hologr\'aficas:
Control simult\'aneo de posici\'on y rotaci\'on}}

\subsubsection*{Ariel Levenson: Coherent and Dynamic Nonlinear Interactions in
2D Photonic Crystal single and coupled nanocavities}

Coupling light resonantly into a nanocavity mode is a
rather difficult task, when accomplished, new avenues are open to
efficiently produce nonlinear coherent interactions. We discuss recent
results on optical bistability, excitability, symmetry breaking and
slow light in semiconductor L3 Photonic Crystal nanocavities and
coupled nanocavities.

\subsubsection*{F\'elix Requejo: \textit{Investigaciones en Nanociencia con luz
de sincrotr\'on: analisis de propiedades fundamentales en nanomateriales}}